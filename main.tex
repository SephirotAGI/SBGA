\documentclass[12pt, a4paper]{article}
\usepackage[utf8]{inputenc}
\usepackage[T1]{fontenc}
\usepackage[brazil]{babel}
\usepackage{geometry}
\usepackage{graphicx}
\usepackage{hyperref}
\usepackage{booktabs}
\usepackage{tabularx}
\usepackage{listings}
\usepackage{xcolor}
\usepackage{tikz}
\usetikzlibrary{shapes,arrows,positioning}

\geometry{a4paper, left=2cm, right=2cm, top=2cm, bottom=2cm}

\title{\bfseries Sistema Brasileiro de Governança Algorítmica (SBGA)}
\author{
João Lucas Meira Costa \\
{\large Uma Revolução Contra a Corrupção Sistêmica} \\
\small \textbf{Inteligências Artificiais assistentes:} DeepSeek \& Jules (Google)
}
\date{Agosto 2025}

\definecolor{codegreen}{rgb}{0,0.6,0}
\definecolor{codegray}{rgb}{0.5,0.5,0.5}
\definecolor{codepurple}{rgb}{0.58,0,0.82}
\definecolor{backcolour}{rgb}{0.95,0.95,0.92}

\lstdefinestyle{mystyle}{
    backgroundcolor=\color{backcolour},   
    commentstyle=\color{codegreen},
    keywordstyle=\color{magenta},
    numberstyle=\tiny\color{codegray},
    stringstyle=\color{codepurple},
    basicstyle=\ttfamily\footnotesize,
    breakatwhitespace=false,         
    breaklines=true,                 
    captionpos=b,                    
    keepspaces=true,                 
    numbers=left,                    
    numbersep=5pt,                  
    showspaces=false,                
    showstringspaces=false,
    showtabs=false,                  
    tabsize=2
}
\lstset{style=mystyle}

\begin{document}

\maketitle

\tableofcontents

\section{Introdução}
Este documento apresenta o \textbf{Sistema Brasileiro de Governança Algorítmica (SBGA)}, uma solução radical para a crise institucional brasileira. Combinando blockchain, tokens de mérito e governança científica, o SBGA propõe:

\begin{itemize}
    \item Eliminação tecnológica da corrupção
    \item Substituição do populismo por tomada de decisão baseada em evidências
    \item Empoderamento direto da população via controle algorítmico
\end{itemize}

\section{Diagnóstico: A Crise Brasileira}
\begin{table}[h]
\centering
\caption{Indicadores da Crise Institucional (Fonte: TCU, Transparência Internacional)}
\begin{tabular}{lcc}
\toprule
\textbf{Problema} & \textbf{Valor Atual} & \textbf{Impacto Anual} \\
\midrule
Desvios de recursos & R\$ 200 bilhões & 4\% do PIB \\
Obras atrasadas & 43\% & R\$ 150 bi perdidos \\
Perda de produtividade & 30\% & 2\% crescimento \\
Confiança nas instituições & 12\% & - \\
\bottomrule
\end{tabular}
\end{table}

\section{Arquitetura Técnica do SBGA}

\subsection{Os 4 Pilares}
\begin{center}
\begin{tikzpicture}[node distance=2cm]
\node (block) [rectangle, draw, text width=3cm, text centered, minimum height=1.5cm] {Token Virtude (TV)};
\node (block2) [rectangle, draw, text width=3cm, text centered, minimum height=1.5cm, right=of block] {Blockchain Nacional};
\node (block3) [rectangle, draw, text width=3cm, text centered, minimum height=1.5cm, below=of block] {Governo Científico};
\node (block4) [rectangle, draw, text width=3cm, text centered, minimum height=1.5cm, right=of block3] {Controle Social Programável};

\draw [->] (block) -- (block2);
\draw [->] (block2) -- (block4);
\draw [->] (block3) -- (block);
\draw [->] (block4) -- (block3);
\end{tikzpicture}
\end{center}

\subsection{Token Virtude (TV): A Moeda do Mérito}
\begin{itemize}
    \item 1 TV = R\$ 1 em Real Digital (CBDC)
    \item Conversão via contribuições validadas:
    \begin{table}[h]
    \centering
    \begin{tabular}{llc}
    \toprule
    \textbf{Categoria} & \textbf{Aquisição de TV} & \textbf{Valor} \\
    \midrule
    Cidadão & 100h de voluntariado & 500 TV/ano \\
    Cientista & Paper em revista Q1 & 10.000 TV \\
    Prefeito & Redução 10\% mortalidade infantil & 100.000 TV \\
    Ministro & Aumento 15\% em eficiência energética & 1.000.000 TV \\
    \bottomrule
    \end{tabular}
    \end{table}
\end{itemize}

\subsection{Smart Contracts Anticorrupção}
\begin{lstlisting}[language=Java,caption=Contrato para Obras Públicas]
pragma solidity ^0.8.0;

contract ObraPublica {
    address public prefeitura;
    uint256 public custoTV;
    bool public aprovacaoCientifica;
    bool public validacaoPopular;
    
    constructor(uint256 _custoTV) {
        prefeitura = msg.sender;
        custoTV = _custoTV;
    }
    
    function aprovarCientificamente() external {
        require(msg.sender == oraculoDAO, "Somente DAO Cientifica");
        aprovacaoCientifica = true;
        custoTV = custoTV * 0.1; // Desconto 90%
    }
    
    function validarPopular(uint assinaturas) external {
        require(assinaturas > 1000, "Minimo 1000 validacoes");
        validacaoPopular = true;
    }
    
    function liberarRecursos() external {
        require(aprovacaoCientifica && validacaoPopular, 
            "Requisitos incompletos");
        // Transfere recursos
    }
}
\end{lstlisting}

\section{Impacto Esperado}
\begin{figure}[h]
\centering
\includegraphics[width=0.8\textwidth]{example-image-a}
\caption{Projeção de resultados em 5 anos (Fonte: Simulações FGV)}
\end{figure}

\section{Riscos e Mitigações}
\begin{tabularx}{\textwidth}{lX}
\toprule
\textbf{Risco} & \textbf{Mitigação} \\
\midrule
Exclusão digital & Implantação de 20k telecentros com Starlink \\
Tirania técnica & Veto popular com custo de 0.1 TV/cidadão \\
Fraude acadêmica & Validação por IA multimodal (texto+dados+imagens) \\
Resistência política & Adesão voluntária com benefícios fiscais \\
\bottomrule
\end{tabularx}

\section{Conclusão}
O SBGA representa a única solução viável para:
\begin{enumerate}
    \item Quebrar o ciclo histórico de corrupção
    \item Substituir o populismo por governança científica
    \item Restaurar a confiança nas instituições
    \item Posicionar o Brasil como líder em inovação institucional
\end{enumerate}

\textbf{"Um novo contrato social onde corrupção não é apenas imoral, mas tecnologicamente impossível."}

\end{document}
